\haddockmoduleheading{Data.Char}
\label{module:Data.Char}
\haddockbeginheader
{\haddockverb\begin{verbatim}
module Data.Char (
    Char,  String,  isControl,  isSpace,  isLower,  isUpper,  isAlpha, 
    isAlphaNum,  isPrint,  isDigit,  isOctDigit,  isHexDigit,  isLetter, 
    isMark,  isNumber,  isPunctuation,  isSymbol,  isSeparator,  isAscii, 
    isLatin1,  isAsciiUpper,  isAsciiLower, 
    GeneralCategory(UppercaseLetter,
                    LowercaseLetter,
                    TitlecaseLetter,
                    ModifierLetter,
                    OtherLetter,
                    NonSpacingMark,
                    SpacingCombiningMark,
                    EnclosingMark,
                    DecimalNumber,
                    LetterNumber,
                    OtherNumber,
                    ConnectorPunctuation,
                    DashPunctuation,
                    OpenPunctuation,
                    ClosePunctuation,
                    InitialQuote,
                    FinalQuote,
                    OtherPunctuation,
                    MathSymbol,
                    CurrencySymbol,
                    ModifierSymbol,
                    OtherSymbol,
                    Space,
                    LineSeparator,
                    ParagraphSeparator,
                    Control,
                    Format,
                    Surrogate,
                    PrivateUse,
                    NotAssigned), 
    generalCategory,  toUpper,  toLower,  toTitle,  digitToInt,  intToDigit, 
    ord,  chr,  showLitChar,  lexLitChar,  readLitChar
  ) where\end{verbatim}}
\haddockendheader

\section{Characters and strings
}
\begin{haddockdesc}
\item[\begin{tabular}{@{}l}
data\ Char
\end{tabular}]\haddockbegindoc
The character type \haddockid{Char} is an enumeration whose values represent
Unicode (or equivalently ISO/IEC 10646) characters
(see \url{http://www.unicode.org/} for details).
This set extends the ISO 8859-1 (Latin-1) character set
(the first 256 charachers), which is itself an extension of the ASCII
character set (the first 128 characters).
A character literal in Haskell has type \haddockid{Char}.
\par
To convert a \haddockid{Char} to or from the corresponding \haddockid{Int} value defined
by Unicode, use \haddocktt{Prelude.toEnum} and \haddocktt{Prelude.fromEnum} from the
\haddocktt{Prelude.Enum} class respectively (or equivalently \haddocktt{ord} and \haddocktt{chr}).
\par

\end{haddockdesc}
\begin{haddockdesc}
\item[\begin{tabular}{@{}l}
instance\ Bounded\ Char\\instance\ Enum\ Char\\instance\ Eq\ Char\\instance\ Ord\ Char\\instance\ Read\ Char\\instance\ Show\ Char\\instance\ Ix\ Char\\instance\ Storable\ Char
\end{tabular}]
\end{haddockdesc}
\begin{haddockdesc}
\item[\begin{tabular}{@{}l}
type\ String\ =\ {\char 91}Char{\char 93}
\end{tabular}]\haddockbegindoc
A \haddockid{String} is a list of characters.  String constants in Haskell are values
 of type \haddockid{String}.
\par

\end{haddockdesc}
\section{Character classification
}
Unicode characters are divided into letters, numbers, marks,
 punctuation, symbols, separators (including spaces) and others
 (including control characters).
\par

\begin{haddockdesc}
\item[\begin{tabular}{@{}l}
isControl\ ::\ Char\ ->\ Bool
\end{tabular}]\haddockbegindoc
Selects control characters, which are the non-printing characters of
 the Latin-1 subset of Unicode.
\par

\end{haddockdesc}
\begin{haddockdesc}
\item[\begin{tabular}{@{}l}
isSpace\ ::\ Char\ ->\ Bool
\end{tabular}]\haddockbegindoc
Returns \haddockid{True} for any Unicode space character, and the control
 characters \haddocktt{{\char '134}t}, \haddocktt{{\char '134}n}, \haddocktt{{\char '134}r}, \haddocktt{{\char '134}f}, \haddocktt{{\char '134}v}.
\par

\end{haddockdesc}
\begin{haddockdesc}
\item[\begin{tabular}{@{}l}
isLower\ ::\ Char\ ->\ Bool
\end{tabular}]\haddockbegindoc
Selects lower-case alphabetic Unicode characters (letters).
\par

\end{haddockdesc}
\begin{haddockdesc}
\item[\begin{tabular}{@{}l}
isUpper\ ::\ Char\ ->\ Bool
\end{tabular}]\haddockbegindoc
Selects upper-case or title-case alphabetic Unicode characters (letters).
 Title case is used by a small number of letter ligatures like the
 single-character form of \emph{Lj}.
\par

\end{haddockdesc}
\begin{haddockdesc}
\item[\begin{tabular}{@{}l}
isAlpha\ ::\ Char\ ->\ Bool
\end{tabular}]\haddockbegindoc
Selects alphabetic Unicode characters (lower-case, upper-case and
 title-case letters, plus letters of caseless scripts and modifiers letters).
 This function is equivalent to \haddocktt{Data.Char.isLetter}.
\par

\end{haddockdesc}
\begin{haddockdesc}
\item[\begin{tabular}{@{}l}
isAlphaNum\ ::\ Char\ ->\ Bool
\end{tabular}]\haddockbegindoc
Selects alphabetic or numeric digit Unicode characters.
\par
Note that numeric digits outside the ASCII range are selected by this
 function but not by \haddockid{isDigit}.  Such digits may be part of identifiers
 but are not used by the printer and reader to represent numbers.
\par

\end{haddockdesc}
\begin{haddockdesc}
\item[\begin{tabular}{@{}l}
isPrint\ ::\ Char\ ->\ Bool
\end{tabular}]\haddockbegindoc
Selects printable Unicode characters
 (letters, numbers, marks, punctuation, symbols and spaces).
\par

\end{haddockdesc}
\begin{haddockdesc}
\item[\begin{tabular}{@{}l}
isDigit\ ::\ Char\ ->\ Bool
\end{tabular}]\haddockbegindoc
Selects ASCII digits, i.e. \haddocktt{'0'}..\haddocktt{'9'}.
\par

\end{haddockdesc}
\begin{haddockdesc}
\item[\begin{tabular}{@{}l}
isOctDigit\ ::\ Char\ ->\ Bool
\end{tabular}]\haddockbegindoc
Selects ASCII octal digits, i.e. \haddocktt{'0'}..\haddocktt{'7'}.
\par

\end{haddockdesc}
\begin{haddockdesc}
\item[\begin{tabular}{@{}l}
isHexDigit\ ::\ Char\ ->\ Bool
\end{tabular}]\haddockbegindoc
Selects ASCII hexadecimal digits,
 i.e. \haddocktt{'0'}..\haddocktt{'9'}, \haddocktt{'a'}..\haddocktt{'f'}, \haddocktt{'A'}..\haddocktt{'F'}.
\par

\end{haddockdesc}
\begin{haddockdesc}
\item[\begin{tabular}{@{}l}
isLetter\ ::\ Char\ ->\ Bool
\end{tabular}]\haddockbegindoc
Selects alphabetic Unicode characters (lower-case, upper-case and
 title-case letters, plus letters of caseless scripts and modifiers letters).
 This function is equivalent to \haddocktt{Data.Char.isAlpha}.
\par

\end{haddockdesc}
\begin{haddockdesc}
\item[\begin{tabular}{@{}l}
isMark\ ::\ Char\ ->\ Bool
\end{tabular}]\haddockbegindoc
Selects Unicode mark characters, e.g. accents and the like, which
 combine with preceding letters.
\par

\end{haddockdesc}
\begin{haddockdesc}
\item[\begin{tabular}{@{}l}
isNumber\ ::\ Char\ ->\ Bool
\end{tabular}]\haddockbegindoc
Selects Unicode numeric characters, including digits from various
 scripts, Roman numerals, etc.
\par

\end{haddockdesc}
\begin{haddockdesc}
\item[\begin{tabular}{@{}l}
isPunctuation\ ::\ Char\ ->\ Bool
\end{tabular}]\haddockbegindoc
Selects Unicode punctuation characters, including various kinds
 of connectors, brackets and quotes.
\par

\end{haddockdesc}
\begin{haddockdesc}
\item[\begin{tabular}{@{}l}
isSymbol\ ::\ Char\ ->\ Bool
\end{tabular}]\haddockbegindoc
Selects Unicode symbol characters, including mathematical and
 currency symbols.
\par

\end{haddockdesc}
\begin{haddockdesc}
\item[\begin{tabular}{@{}l}
isSeparator\ ::\ Char\ ->\ Bool
\end{tabular}]\haddockbegindoc
Selects Unicode space and separator characters.
\par

\end{haddockdesc}
\subsection{Subranges
}
\begin{haddockdesc}
\item[\begin{tabular}{@{}l}
isAscii\ ::\ Char\ ->\ Bool
\end{tabular}]\haddockbegindoc
Selects the first 128 characters of the Unicode character set,
 corresponding to the ASCII character set.
\par

\end{haddockdesc}
\begin{haddockdesc}
\item[\begin{tabular}{@{}l}
isLatin1\ ::\ Char\ ->\ Bool
\end{tabular}]\haddockbegindoc
Selects the first 256 characters of the Unicode character set,
 corresponding to the ISO 8859-1 (Latin-1) character set.
\par

\end{haddockdesc}
\begin{haddockdesc}
\item[\begin{tabular}{@{}l}
isAsciiUpper\ ::\ Char\ ->\ Bool
\end{tabular}]\haddockbegindoc
Selects ASCII upper-case letters,
 i.e. characters satisfying both \haddockid{isAscii} and \haddockid{isUpper}.
\par

\end{haddockdesc}
\begin{haddockdesc}
\item[\begin{tabular}{@{}l}
isAsciiLower\ ::\ Char\ ->\ Bool
\end{tabular}]\haddockbegindoc
Selects ASCII lower-case letters,
 i.e. characters satisfying both \haddockid{isAscii} and \haddockid{isLower}.
\par

\end{haddockdesc}
\subsection{Unicode general categories
}
\begin{haddockdesc}
\item[\begin{tabular}{@{}l}
data\ GeneralCategory
\end{tabular}]\haddockbegindoc
\haddockbeginconstrs
\haddockdecltt{=} & \haddockdecltt{UppercaseLetter} & Lu: Letter, Uppercase
 \\
\haddockdecltt{|} & \haddockdecltt{LowercaseLetter} & Ll: Letter, Lowercase
 \\
\haddockdecltt{|} & \haddockdecltt{TitlecaseLetter} & Lt: Letter, Titlecase
 \\
\haddockdecltt{|} & \haddockdecltt{ModifierLetter} & Lm: Letter, Modifier
 \\
\haddockdecltt{|} & \haddockdecltt{OtherLetter} & Lo: Letter, Other
 \\
\haddockdecltt{|} & \haddockdecltt{NonSpacingMark} & Mn: Mark, Non-Spacing
 \\
\haddockdecltt{|} & \haddockdecltt{SpacingCombiningMark} & Mc: Mark, Spacing Combining
 \\
\haddockdecltt{|} & \haddockdecltt{EnclosingMark} & Me: Mark, Enclosing
 \\
\haddockdecltt{|} & \haddockdecltt{DecimalNumber} & Nd: Number, Decimal
 \\
\haddockdecltt{|} & \haddockdecltt{LetterNumber} & Nl: Number, Letter
 \\
\haddockdecltt{|} & \haddockdecltt{OtherNumber} & No: Number, Other
 \\
\haddockdecltt{|} & \haddockdecltt{ConnectorPunctuation} & Pc: Punctuation, Connector
 \\
\haddockdecltt{|} & \haddockdecltt{DashPunctuation} & Pd: Punctuation, Dash
 \\
\haddockdecltt{|} & \haddockdecltt{OpenPunctuation} & Ps: Punctuation, Open
 \\
\haddockdecltt{|} & \haddockdecltt{ClosePunctuation} & Pe: Punctuation, Close
 \\
\haddockdecltt{|} & \haddockdecltt{InitialQuote} & Pi: Punctuation, Initial quote
 \\
\haddockdecltt{|} & \haddockdecltt{FinalQuote} & Pf: Punctuation, Final quote
 \\
\haddockdecltt{|} & \haddockdecltt{OtherPunctuation} & Po: Punctuation, Other
 \\
\haddockdecltt{|} & \haddockdecltt{MathSymbol} & Sm: Symbol, Math
 \\
\haddockdecltt{|} & \haddockdecltt{CurrencySymbol} & Sc: Symbol, Currency
 \\
\haddockdecltt{|} & \haddockdecltt{ModifierSymbol} & Sk: Symbol, Modifier
 \\
\haddockdecltt{|} & \haddockdecltt{OtherSymbol} & So: Symbol, Other
 \\
\haddockdecltt{|} & \haddockdecltt{Space} & Zs: Separator, Space
 \\
\haddockdecltt{|} & \haddockdecltt{LineSeparator} & Zl: Separator, Line
 \\
\haddockdecltt{|} & \haddockdecltt{ParagraphSeparator} & Zp: Separator, Paragraph
 \\
\haddockdecltt{|} & \haddockdecltt{Control} & Cc: Other, Control
 \\
\haddockdecltt{|} & \haddockdecltt{Format} & Cf: Other, Format
 \\
\haddockdecltt{|} & \haddockdecltt{Surrogate} & Cs: Other, Surrogate
 \\
\haddockdecltt{|} & \haddockdecltt{PrivateUse} & Co: Other, Private Use
 \\
\haddockdecltt{|} & \haddockdecltt{NotAssigned} & Cn: Other, Not Assigned
 \\
\end{tabulary}\par
Unicode General Categories (column 2 of the UnicodeData table)
 in the order they are listed in the Unicode standard.
\par

\end{haddockdesc}
\begin{haddockdesc}
\item[\begin{tabular}{@{}l}
instance\ Bounded\ GeneralCategory\\instance\ Enum\ GeneralCategory\\instance\ Eq\ GeneralCategory\\instance\ Ord\ GeneralCategory\\instance\ Read\ GeneralCategory\\instance\ Show\ GeneralCategory\\instance\ Ix\ GeneralCategory
\end{tabular}]
\end{haddockdesc}
\begin{haddockdesc}
\item[\begin{tabular}{@{}l}
generalCategory\ ::\ Char\ ->\ GeneralCategory
\end{tabular}]\haddockbegindoc
The Unicode general category of the character.
\par

\end{haddockdesc}
\section{Case conversion
}
\begin{haddockdesc}
\item[\begin{tabular}{@{}l}
toUpper\ ::\ Char\ ->\ Char
\end{tabular}]\haddockbegindoc
Convert a letter to the corresponding upper-case letter, if any.
 Any other character is returned unchanged.
\par

\end{haddockdesc}
\begin{haddockdesc}
\item[\begin{tabular}{@{}l}
toLower\ ::\ Char\ ->\ Char
\end{tabular}]\haddockbegindoc
Convert a letter to the corresponding lower-case letter, if any.
 Any other character is returned unchanged.
\par

\end{haddockdesc}
\begin{haddockdesc}
\item[\begin{tabular}{@{}l}
toTitle\ ::\ Char\ ->\ Char
\end{tabular}]\haddockbegindoc
Convert a letter to the corresponding title-case or upper-case
 letter, if any.  (Title case differs from upper case only for a small
 number of ligature letters.)
 Any other character is returned unchanged.
\par

\end{haddockdesc}
\section{Single digit characters
}
\begin{haddockdesc}
\item[\begin{tabular}{@{}l}
digitToInt\ ::\ Char\ ->\ Int
\end{tabular}]\haddockbegindoc
Convert a single digit \haddockid{Char} to the corresponding \haddockid{Int}.  
 This function fails unless its argument satisfies \haddockid{isHexDigit},
 but recognises both upper and lower-case hexadecimal digits
 (i.e. \haddocktt{'0'}..\haddocktt{'9'}, \haddocktt{'a'}..\haddocktt{'f'}, \haddocktt{'A'}..\haddocktt{'F'}).
\par

\end{haddockdesc}
\begin{haddockdesc}
\item[\begin{tabular}{@{}l}
intToDigit\ ::\ Int\ ->\ Char
\end{tabular}]\haddockbegindoc
Convert an \haddockid{Int} in the range \haddocktt{0}..\haddocktt{15} to the corresponding single
 digit \haddockid{Char}.  This function fails on other inputs, and generates
 lower-case hexadecimal digits.
\par

\end{haddockdesc}
\section{Numeric representations
}
\begin{haddockdesc}
\item[\begin{tabular}{@{}l}
ord\ ::\ Char\ ->\ Int
\end{tabular}]\haddockbegindoc
The \haddocktt{Prelude.fromEnum} method restricted to the type \haddocktt{Data.Char.Char}.
\par

\end{haddockdesc}
\begin{haddockdesc}
\item[\begin{tabular}{@{}l}
chr\ ::\ Int\ ->\ Char
\end{tabular}]\haddockbegindoc
The \haddocktt{Prelude.toEnum} method restricted to the type \haddocktt{Data.Char.Char}.
\par

\end{haddockdesc}
\section{String representations
}
\begin{haddockdesc}
\item[\begin{tabular}{@{}l}
showLitChar\ ::\ Char\ ->\ ShowS
\end{tabular}]\haddockbegindoc
Convert a character to a string using only printable characters,
 using Haskell source-language escape conventions.  For example:
\par
\begin{quote}
{\haddockverb\begin{verbatim}
 showLitChar '\n' s  =  "\\n" ++ s
\end{verbatim}}
\end{quote}

\end{haddockdesc}
\begin{haddockdesc}
\item[\begin{tabular}{@{}l}
lexLitChar\ ::\ ReadS\ String
\end{tabular}]\haddockbegindoc
Read a string representation of a character, using Haskell
 source-language escape conventions.  For example:
\par
\begin{quote}
{\haddockverb\begin{verbatim}
 lexLitChar  "\\nHello"  =  [("\\n", "Hello")]
\end{verbatim}}
\end{quote}

\end{haddockdesc}
\begin{haddockdesc}
\item[\begin{tabular}{@{}l}
readLitChar\ ::\ ReadS\ Char
\end{tabular}]\haddockbegindoc
Read a string representation of a character, using Haskell
 source-language escape conventions, and convert it to the character
 that it encodes.  For example:
\par
\begin{quote}
{\haddockverb\begin{verbatim}
 readLitChar "\\nHello"  =  [('\n', "Hello")]
\end{verbatim}}
\end{quote}

\end{haddockdesc}