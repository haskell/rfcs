\haddockmoduleheading{Numeric}
\label{module:Numeric}
\haddockbeginheader
{\haddockverb\begin{verbatim}
module Numeric (
    showSigned,  showIntAtBase,  showInt,  showHex,  showOct,  showEFloat, 
    showFFloat,  showGFloat,  showFloat,  floatToDigits,  readSigned,  readInt, 
    readDec,  readOct,  readHex,  readFloat,  lexDigits,  fromRat
  ) where\end{verbatim}}
\haddockendheader

\section{Showing
}
\begin{haddockdesc}
\item[\begin{tabular}{@{}l}
showSigned
\end{tabular}]\haddockbegindoc
\haddockbeginargs
\haddockdecltt{::} & \haddockdecltt{Real a} \\
                     \haddockdecltt{=>} & \haddockdecltt{(a
                                                          -> ShowS)} & a function that can show unsigned values
 \\
                                                                       \haddockdecltt{->} & \haddockdecltt{Int} & the precedence of the enclosing context
 \\
                                                                                                                  \haddockdecltt{->} & \haddockdecltt{a} & the value to show
 \\
                                                                                                                                                           \haddockdecltt{->} & \haddockdecltt{ShowS} & \\
\end{tabulary}\par
Converts a possibly-negative \haddockid{Real} value to a string.
\par

\end{haddockdesc}
\begin{haddockdesc}
\item[\begin{tabular}{@{}l}
showIntAtBase\ ::\ Integral\ a\ =>\ a\ ->\ (Int\ ->\ Char)\ ->\ a\ ->\ ShowS
\end{tabular}]\haddockbegindoc
Shows a \emph{non-negative} \haddockid{Integral} number using the base specified by the
 first argument, and the character representation specified by the second.
\par

\end{haddockdesc}
\begin{haddockdesc}
\item[\begin{tabular}{@{}l}
showInt\ ::\ Integral\ a\ =>\ a\ ->\ ShowS
\end{tabular}]\haddockbegindoc
Show \emph{non-negative} \haddockid{Integral} numbers in base 10.
\par

\end{haddockdesc}
\begin{haddockdesc}
\item[\begin{tabular}{@{}l}
showHex\ ::\ Integral\ a\ =>\ a\ ->\ ShowS
\end{tabular}]\haddockbegindoc
Show \emph{non-negative} \haddockid{Integral} numbers in base 16.
\par

\end{haddockdesc}
\begin{haddockdesc}
\item[\begin{tabular}{@{}l}
showOct\ ::\ Integral\ a\ =>\ a\ ->\ ShowS
\end{tabular}]\haddockbegindoc
Show \emph{non-negative} \haddockid{Integral} numbers in base 8.
\par

\end{haddockdesc}
\begin{haddockdesc}
\item[\begin{tabular}{@{}l}
showEFloat\ ::\ RealFloat\ a\ =>\ Maybe\ Int\ ->\ a\ ->\ ShowS
\end{tabular}]\haddockbegindoc
Show a signed \haddockid{RealFloat} value
 using scientific (exponential) notation (e.g. \haddocktt{2.45e2}, \haddocktt{1.5e-3}).
\par
In the call \haddocktt{showEFloat\ digs\ val}, if \haddocktt{digs} is \haddockid{Nothing},
 the value is shown to full precision; if \haddocktt{digs} is \haddocktt{Just\ d},
 then at most \haddocktt{d} digits after the decimal point are shown.
\par

\end{haddockdesc}
\begin{haddockdesc}
\item[\begin{tabular}{@{}l}
showFFloat\ ::\ RealFloat\ a\ =>\ Maybe\ Int\ ->\ a\ ->\ ShowS
\end{tabular}]\haddockbegindoc
Show a signed \haddockid{RealFloat} value
 using standard decimal notation (e.g. \haddocktt{245000}, \haddocktt{0.0015}).
\par
In the call \haddocktt{showFFloat\ digs\ val}, if \haddocktt{digs} is \haddockid{Nothing},
 the value is shown to full precision; if \haddocktt{digs} is \haddocktt{Just\ d},
 then at most \haddocktt{d} digits after the decimal point are shown.
\par

\end{haddockdesc}
\begin{haddockdesc}
\item[\begin{tabular}{@{}l}
showGFloat\ ::\ RealFloat\ a\ =>\ Maybe\ Int\ ->\ a\ ->\ ShowS
\end{tabular}]\haddockbegindoc
Show a signed \haddockid{RealFloat} value
 using standard decimal notation for arguments whose absolute value lies 
 between \haddocktt{0.1} and \haddocktt{9,999,999}, and scientific notation otherwise.
\par
In the call \haddocktt{showGFloat\ digs\ val}, if \haddocktt{digs} is \haddockid{Nothing},
 the value is shown to full precision; if \haddocktt{digs} is \haddocktt{Just\ d},
 then at most \haddocktt{d} digits after the decimal point are shown.
\par

\end{haddockdesc}
\begin{haddockdesc}
\item[\begin{tabular}{@{}l}
showFloat\ ::\ RealFloat\ a\ =>\ a\ ->\ ShowS
\end{tabular}]\haddockbegindoc
Show a signed \haddockid{RealFloat} value to full precision
 using standard decimal notation for arguments whose absolute value lies 
 between \haddocktt{0.1} and \haddocktt{9,999,999}, and scientific notation otherwise.
\par

\end{haddockdesc}
\begin{haddockdesc}
\item[\begin{tabular}{@{}l}
floatToDigits\ ::\ RealFloat\ a\ =>\ Integer\ ->\ a\ ->\ ({\char 91}Int{\char 93},\ Int)
\end{tabular}]\haddockbegindoc
\haddockid{floatToDigits} takes a base and a non-negative \haddockid{RealFloat} number,
 and returns a list of digits and an exponent. 
 In particular, if \haddocktt{x>=0}, and
\par
\begin{quote}
{\haddockverb\begin{verbatim}
 floatToDigits base x = ([d1,d2,...,dn], e)
\end{verbatim}}
\end{quote}
then
\par
\begin{enumerate}
\item 
\begin{quote}
{\haddockverb\begin{verbatim}
n >= 1\end{verbatim}}
\end{quote}

\item 
\begin{quote}
{\haddockverb\begin{verbatim}
x = 0.d1d2...dn * (base**e)\end{verbatim}}
\end{quote}

\item 
\begin{quote}
{\haddockverb\begin{verbatim}
0 <= di <= base-1\end{verbatim}}
\end{quote}

\end{enumerate}

\end{haddockdesc}
\section{Reading
}
\emph{NB:} \haddockid{readInt} is the 'dual' of \haddockid{showIntAtBase},
 and \haddockid{readDec} is the `dual' of \haddockid{showInt}.
 The inconsistent naming is a historical accident.
\par

\begin{haddockdesc}
\item[\begin{tabular}{@{}l}
readSigned\ ::\ Real\ a\ =>\ ReadS\ a\ ->\ ReadS\ a
\end{tabular}]\haddockbegindoc
Reads a \emph{signed} \haddockid{Real} value, given a reader for an unsigned value.
\par

\end{haddockdesc}
\begin{haddockdesc}
\item[\begin{tabular}{@{}l}
readInt
\end{tabular}]\haddockbegindoc
\haddockbeginargs
\haddockdecltt{::} & \haddockdecltt{Num a} \\
                     \haddockdecltt{=>} & \haddockdecltt{a} & the base
 \\
                                                              \haddockdecltt{->} & \haddockdecltt{(Char
                                                                                                   -> Bool)} & a predicate distinguishing valid digits in this base
 \\
                                                                                                               \haddockdecltt{->} & \haddockdecltt{(Char
                                                                                                                                                    -> Int)} & a function converting a valid digit character to an \haddockid{Int} \\
                                                                                                                                                               \haddockdecltt{->} & \haddockdecltt{ReadS a} & \\
\end{tabulary}\par
Reads an \emph{unsigned} \haddockid{Integral} value in an arbitrary base.
\par

\end{haddockdesc}
\begin{haddockdesc}
\item[\begin{tabular}{@{}l}
readDec\ ::\ Num\ a\ =>\ ReadS\ a
\end{tabular}]\haddockbegindoc
Read an unsigned number in decimal notation.
\par

\end{haddockdesc}
\begin{haddockdesc}
\item[\begin{tabular}{@{}l}
readOct\ ::\ Num\ a\ =>\ ReadS\ a
\end{tabular}]\haddockbegindoc
Read an unsigned number in octal notation.
\par

\end{haddockdesc}
\begin{haddockdesc}
\item[\begin{tabular}{@{}l}
readHex\ ::\ Num\ a\ =>\ ReadS\ a
\end{tabular}]\haddockbegindoc
Read an unsigned number in hexadecimal notation.
 Both upper or lower case letters are allowed.
\par

\end{haddockdesc}
\begin{haddockdesc}
\item[\begin{tabular}{@{}l}
readFloat\ ::\ RealFrac\ a\ =>\ ReadS\ a
\end{tabular}]\haddockbegindoc
Reads an \emph{unsigned} \haddockid{RealFrac} value,
 expressed in decimal scientific notation.
\par

\end{haddockdesc}
\begin{haddockdesc}
\item[\begin{tabular}{@{}l}
lexDigits\ ::\ ReadS\ String
\end{tabular}]\haddockbegindoc
Reads a non-empty string of decimal digits.
\par

\end{haddockdesc}
\section{Miscellaneous
}
\begin{haddockdesc}
\item[\begin{tabular}{@{}l}
fromRat\ ::\ RealFloat\ a\ =>\ Rational\ ->\ a
\end{tabular}]\haddockbegindoc
Converts a \haddockid{Rational} value into any type in class \haddockid{RealFloat}.
\par

\end{haddockdesc}